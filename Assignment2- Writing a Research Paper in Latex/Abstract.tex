\begin{abstract}
Power side-channel attacks exploit variations in a processor’s energy consumption to extract secret information, posing a significant threat to secure computation in modern systems. In this work, we investigate the current feasibility of such attacks in light of industry-deployed mitigations, focusing on Intel’s post-PLATYPUS microcode updates and their effect on attack surface. We evaluate the persistence of side-channel leakage on an updated x86 platform by reproducing and extending PLATYPUS-style analyses, including instruction benchmarking, KASLR derandomization, and attacks on RSA exponentiation within an Intel SGX enclave.
Our findings confirm that disabling TSX via microcode effectively eliminates KASLR derandomization using power analysis. However, energy measurements obtained from the RAPL interface remain sufficiently sensitive to differentiate instruction types and operand Hamming weights, exposing a residual leakage channel. These results highlight that, despite microcode-level countermeasures, RAPL-based power side channels still threaten cryptographic operations in trusted execution environments. Ongoing research is necessary to develop robust defenses and accurately assess residual vulnerabilities in real-world platforms.\end{abstract}