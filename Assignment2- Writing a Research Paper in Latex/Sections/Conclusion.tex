\section{Conclusion}
This work presents a systematic investigation of the modern attack surface posed by
software-based power side channels on post-mitigation Intel x86 platforms, with a focus on the efficacy of vendor-deployed countermeasures following the publication of the PLATYPUS attack. Through detailed replication and extension of foundational experiments, we have confirmed that while certain attack vectors—most notably TSX-assisted KASLR derandomization—are effectively neutralized by recent microcode updates, other forms of leakage persist.
Specifically, we demonstrated that, even in the presence of energy obfuscation and
access restrictions, the Intel RAPL interface continues to expose coarse-grained but distinguishable power signatures for different instructions and data values. Instruction-level energy differences and operand Hamming weights remain measurable, with implications for the security of cryptographic operations. Our experiments further validated the continued vulnerability of RSA modular exponentiation within SGX enclaves: power usage remains highly dependent on secret key material, and RAPL can differentiate between low- and high-entropy keys even after all countermeasures are deployed.
These findings reinforce that software-accessible power side channels, though blunted, cannot be fully eliminated by microcode-based noise injection or interface gating alone. Ongoing research and a careful re-evaluation of trusted execution environment threat models are needed to close the remaining gaps.